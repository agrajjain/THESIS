{\msabstract{Prof. Farshad Khorrami}}
{

This thesis presents a study on implementing Simultaneous Localization and Mapping(SLAM) on an Integrated Mobile Platform(IMP). The IMP has been designed to have autonomous navigation capability.The modular design enables testing of a variety of navigation algorithms,control designs and SLAM. The design and construction of IMP is detailed in Chapter~\ref{cha:Platform } of this thesis. A common challenge in most implementations of autonomous navigation is localization and mapping. Towards this end, an Extended Kalman Filter(EKF) based SLAM algorithm is introduced and implemented. The SLAM implementation is modular and designed towards easy implementation of different algorithms. The thesis includes discussion on the mathematical background as well as insights into the practical implementation of SLAM on a ground vehicle such as IMP. 

Two feature detectors are implemented in the SLAM algorithm. The feature detectors extract features predicted to be in the environment using data acquired from a 2D scanning laser range finder(LIDAR). A simplistic feature extraction method to detect point features in the environment is discussed along with its results and drawbacks. Next, linear feature extraction algorithms are discussed, both conceptually and practically. The algorithms discussed are utilized in the modular implementation of SLAM.The thesis discusses insights about the effectiveness and robustness of the feature extraction techniques and their efficacy in SLAM implementation are presented and discussed in this thesis.Experimental results for feature extraction and SLAM, in both controlled and uncontrolled indoor environments are analyzed.
}
{\endmsabstract}