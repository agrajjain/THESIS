\documentstyle[epic,eepic,graphicx,cite,mathtools,caption,subcaption,algorithm,algorithmicx,algpseudocode,longtable]{./styleDocs/poly3}
%\mathindent 0pt
\input{epsf}

\renewcommand{\baselinestretch}{1.6}
\newtheorem{theorem}{Theorem}[chapter]
\newcommand{\btheorem}{\begin{theorem}\rm}
\newcommand{\etheorem}{$\diamond$\end{theorem}}
\newtheorem{definition}{Definition}[chapter]
\newcommand{\bdefn}{\begin{definition}\rm}
\newcommand{\edefn}{\end{definition}}
\newtheorem{lemma}{Lemma}[chapter]
\newtheorem{remark}{Remark}[chapter]
\newcommand{\bremark}{\begin{remark}\rm}
\newcommand{\eremark}{\end{remark}}
\newtheorem{example}{Example}[chapter]
\newcommand{\bexample}{\begin{example}\rm}
\newcommand{\eexample}{\end{example}}
\newtheorem{assumption}{Assumption}[chapter]
\newcommand{\bassump}{\begin{assumption}\rm}
\newcommand{\eassump}{\end{assumption}}
\newcommand{\ekf}{Extended Kalman Filter }
\newcommand{\slam}{Simultaneous Localization and Mapping }
\newcommand{\imp}{Integrated Mobile Platform }
\graphicspath{{./Images/Encoder/}, {./Images/Spike/}, {./Images/Ransac/},{./Images/Platform/}}
\newtagform{defaultoveride}{}{}
\usetagform{defaultoveride}
\begin{document}
\title{Study on various implementations of Extended Kalman Filter based Simultaneous Localization And Mapping}
\author{Agraj Jain}
\committee{ \\ \\ \\ \\ \ \\ \ \\ \ \\Copy Number:}
%\mstitlepage
\topmargin=0.4 in
\textwidth=6.0 in
\textheight=9.0 in
\chapter*{VITA}
\vitaentry{March 1991}{Born}
\vitaentry{June 2009}{Graduated High School}
\vitaentry{May 2013}{Bachelor of Technology in Electrical Engineering}
\vitaentry{May 2015}{Master of Science in Electrical Engineering \textit{(expected)}}
\setcounter{page}{1}
\pagenumbering{roman}
\include{./Roman/publications}
\include{./Roman/acknowledgements} 
\tableofcontents
\listoffigures
\listoftables
\newpage
\setcounter{page}{1}
\pagenumbering{arabic}
\chapter{Introduction}

	Autonomous robots are growing increasingly popular in various fields. They occupy a major fraction of robotics based research today. Autonomous robots take many forms, from mobile robots to fixed manipulators. Even within mobile robots, the diversity in the type of locomotion, the terrain and the targeted use is huge. There are autonomous robots intended for indoor use, for rough terrains, for underwater and also aerial autonomous robots. 
	
	But irrespective of terrain, autonomous mobile robots all have one challenge in common. The ability to map their surroundings while simultaneously figuring out where exactly they are in that map. This concept of Simultaneous Localization and Mapping or SLAM is a vast field of research with numerous algorithms and implementations. Each of these algorithms, have their unique advantages and drawbacks making it impossible to pinpoint a generalized \textit{best algorithm}. 
	
	One of the algorithms used for SLAM is the Extended Kalman Filter. The major advantage of this algorithm is that it is very simple to implement on an on board computer and can be used for real time SLAM applications. Going a bit into EKF SLAM we first use the proprioceptive sensors to get an estimate of our motion. This is called the \textit{Prediction} step. Then we use the exteroceptive sensors to get and idea of how our environment looks like. We try to focus on specific features we know to look for or on generic features. This is called \textit{Feature Extraction}. In subsequent time steps, we compare the environmental features we see, to our previous idea of our environment and use the difference to both improve both our position estimate and our map of our surroundings. Which is the \textit{Correction} step. Hence we can see that, it consists of many sub parts within it which we shall discuss in depth at a later point. Each of these sub parts offer their own challenges and are again focused on the specific application. By making the base algorithm a simple one like EKF, we can plug in different methods and algorithms for each of the subparts without disturbing the rest to give a good idea of their various advantages and disadvantages. 
	
	For testing a basic algorithm like EKF, we use a classic experimental platform which is a differential drive ground vehicle. We use a six wheeled platform with the four exterior wheels being powered and the interior wheels being used for odometry. With this platform we try to map indoor environments such as corridors doors etc. The ground vehicle is also equipped with an autopilot including an Inertial Measurement Unit, an on board computer as well as exteroceptive sensors such as LIDAR and camera.
	
	Going into the subparts of SLAM, the first subpart concentrated on in this thesis is the Feature Extraction. Features can be identifiable features of robust generic features. Features can be extracted both from the camera and from the LIDAR. When dealing with the range data from the LIDAR, a simplistic algorithm is to use rapid changes of the range as features. While this is a simple algorithm used for trying out SLAM for the first time, it has several drawbacks which are subsequently explored. Also modifications for this algorithm are suggested and tried for making it more efficient. 
	
	A number of studies have shown that for indoor SLAM, which is the target environment in this thesis, linear features such as walls and doors are found to give much better results \cite{}. There are a variety of methods\cite{} to find these linear features. A few of them are implemented in python\cite{} and compared with respect to how effective they are for SLAM and their computation time.
	
	Once a good feature extractor which good results for EKF SLAM is implemented, the prediction subpart is analyzed. It essentially consists of using the odometry to get an estimate of the new position of the robot given it's old position. Initially when testing the other algorithms for feature extraction, a simple method of acquiring odometry was used. It was acquired through the encoders on the interior wheels of the mobile platform. Now, once a good feature extractor was implemented, there was scope to explore additional odometry methods such as visual odometry. Such a prediction which does not require the use of encoders, is really useful for application in a wide variety of platforms\cite{} where it is difficult to get odometry like in legged systems or aerial vehicles. 
	
	Once a good visual odometry system is implemented, we combine that with the linear feature extraction based correction to get a good SLAM implementation that is real time implementable. 
	
	
	

	
	
\chapter{Overview of \ekf based Simultaneous Localization and Mapping}
\label{cha:Overview}
\section{Overview of SLAM}
\label{sec:SLAM_parts}
Simultaneous Localization and Mapping, commonly abbreviated as SLAM, is concerned with the problem of a mobile robot building a map of the unknown environment, while at the same time navigating the environment using the map. Slam consists of multiple parts. To give a broad overview, the important parts of SLAM are: 
	\begin{itemize}
		\item \textbf{State Estimation:} Proprioceptive sensors are used to estimate where the robot might be in the map.
		\item \textbf{Landmark Extraction:} Use exteroceptive sensors on the robot to detect prominent features of the environment. These can either be generalized features or specific landmarks about which we have prior information 
		\item \textbf{Data Association:} The landmarks or features detected in the previous step is associated with existing features in the map that has been generated till now.  
		\item \textbf{State Update:} The position of the robot is corrected based on the deviations of the features with the features on the existing map. 
		\item \textbf{Landmark Update:} The positions on the features are also corrected based on the correction in the position of the robot. 
	\end{itemize}

While there are many algorithms for SLAM most contain these basic steps in some form or the other. There are also many ways to solve each of the smaller parts. The purpose of each part is better understood having an understanding of the Extended Kalman Filter. 

\section{Kalman Filter}
\label{sec:KalmanFilter}
The popular Kalman Filter \cite{Kalman1960, WelchandBishop1995} is a consists of mathematical equations that give an efficient computational recursive solution of the least squares problem. The filter has several advantages such as: it supports estimation of the past, present and future states, even when the precise nature of the system is not known.

In general it addresses the problem of trying to estimate the state $ x \in \Re^n $ of a discrete time process described by the linear stochastic equation

\begin{equation}[h]
\label{eq:Kal_1}
x_k = Ax_{k-1} + Bu_k + w_{k-1}
\end{equation}

with a measurement $ z_k \in \Re^m $ which is given by,
\begin{equation}[h]
\label{eq:Kal_2}
z_k=Hx_k+v_k
\end{equation}

The random variables $ w_k and v_k$ represent the process and measurement noise respectively. They are assumed to be independent of each other and have a normal probability distribution functions given by equations \ref{eq:Kal_3} and \ref{eq:Kal_4}
\begin{equation}
\label{eq:Kal_3}
p(w)\approx N(0,Q),
\end{equation} 
\begin{equation}
\label{eq:Kal_4}
p(v) \approx N(0,R)
\end{equation}

In practice, the process and measurement noise covariances, Q and R matrices might change at every time step. As might the matrices A and H. They are assumed constants for now. 

The equations of Kalman filter can be seen as to be in 2 groups. \textit{Prediction} equations and \textit{Correction} equations. The former are responsible for projecting forward in time the current state and error probabilities and the latter are responsible for adjusting the projected estimate by an actual measurement at that time. Hence the Kalman filter estimates a process by a form of feedback control. 

Defining the predicted state estimate as $ \hat{x}^-_k \in \Re^n $ and the corrected estimate to be $ \hat{x}_k \in \Re^n $ the actual equations for the prediction are given by \ref{eq:Kal_5} and \ref{eq:Kal_6} where A and B are from equation \ref{eq:Kal_1} and Q is from equation \ref{eq:Kal_3} \cite{WelchandBishop1995}.
\begin{equation}
\label{eq:Kal_5}
\hat{x}^-_k = A\hat{x}_{k-1}+Bu_k
\end{equation}
\begin{equation}
\label{eq:Kal_6}
P^-_k = AP_{k-1}A^T+Q
\end{equation}

The first step in the correction step is to find a \textit{gain} or \textit{blending factor} that minimizes the error covariance. This is found by equation \ref{eq:Kal_7} \cite{Kalman1960,Maybeck1979,Jacobs1993,Brown2012}. 
The next step is to actually measure the process to get $ z_k $ and generate a better estimate using equations \ref{eq:Kal_8} and \ref{eq:Kal_9} \cite{WelchandBishop1995}.

\begin{equation}
\label{eq:Kal_7}
K_k = P^-_kH^T(HP^-_kH^T+R)^{-1}
\end{equation}
\begin{equation}
\label{eq:Kal_8}
\hat{x}_k = \hat{x}^-_k+K_k(z_k-H\hat{x}^-_k)
\end{equation}
\begin{equation}
\label{eq:Kal_9}
P_k = (I-K_kH)P^-_k
\end{equation} 

After each prediction and correction step, the corrected measurement found out is used as an initial estimate for the prediction step of the next time step. This recursive nature is one of the most appealing feature of Kalman filters. 

\section{Extended Kalman Filter}
\label{sec:EKF}
 In the previous section \ref{sec:KalmanFilter}, the Kalman filter gives us a set of equations to estimate the state $ x \in \Re^n $ of a discrete time process described by a set of \textit{linear} equations. But, most applications including SLAM, consists of systems governed by \textit{non-linear} equations. The Extended Kalman filter is the technique of linearizing a non-linear dynamics around the current mean and covariance for use in a Kalman filter.
 
 Similar to the Taylor series, we can linearize the estimation around the current estimate using the partial derivatives of the process and measurement functions. To do this we slightly modify the equations in section \ref{sec:KalmanFilter}. We again start with the assumption that the process has a state vector $ x \in \Re^n $, but it is described a \textit{non-linear} stochastic difference equation \ref{eq:EKF_1}. This equation relates the state at time step $ x_{k-1}$  and  $x_k $. The measurement $ z \in \Re^m $ is given by equation \ref{eq:EKF_2} where h is also \textit{non-linear}.
 \begin{equation}
 \label{eq:EKF_1}
 x_k = f(x_{k-1},u_k,w_{k-1})
 \end{equation}
 \begin{equation}
 \label{eq:EKF_2}
 z_k = h(x_k,v_k)
 \end{equation}

Here, the random variables $ w_k$  and  $v_k $ are again the process and measurement noise as in equations \ref{eq:Kal_3} ans \ref{eq:Kal_4}. Similar to the section \ref{sec:KalmanFilter}, we have 2 groups of equations. The prediction step is given by equations \ref{eq:EKF_3} and \ref{eq:EKF_4}.
\begin{equation}
 \label{eq:EKF_3}
 \hat{x}^-_k = f(\hat{x}_{k-1},u_k,0)
\end{equation}
\begin{equation}
 \label{eq:EKF_4}
 P^-_k = A_kP_{k-1}A_k^T + W_kQ_{k-1}W^T_k
\end{equation}

Here we can see the first equation is the same as \ref{eq:EKF_1}, but the $ w_k $ variable is replaced by zero as in practice we don ot actually know the noise and $ \hat{x}^-_k $ is just an estimate. The second equation is similar to the Kalman filter equations except here A and W are Jacobian matrices of partial derivatives with respect to x and w respectively. At each time step they are recalculated according to equations \ref{eq:EKF_5} and \ref{eq:EKF_6}

\begin{equation}
\label{eq:EKF_5}
A_{[i,j]} = \frac{\partial f_{[i]}}{\partial x_{[j]}} (\hat{x}_{k-1},u_k,0)
\end{equation}

\begin{equation}
\label{eq:EKF_6}
W_{[i,j]} = \frac{\partial f_{[i]}}{\partial w_{[j]}} (\hat{x}_{k-1},u_k,0)
\end{equation}

As with the basic Kalman filter ,the equations for the correction step given by \ref{eq:EKF_7} to \ref{eq:EKF_9} correct the estimate of the state and covariance based on a measurement $ z $ at time $ k $

\begin{equation}
\label{eq:EKF_7}
K_k = P^-_kH^T_k(_kHP^-_kH^T_k+V_kRV^T_k)^{-1}
\end{equation}
\begin{equation}
\label{eq:EKF_8}
\hat{x}_k = \hat{x}^-_k+K_k(z_k-h(\hat{x}^-_k,0))
\end{equation}
\begin{equation}
\label{eq:EKF_9}
P_k = (I-K_kH_k)P^-_k
\end{equation} 

As in the previous case, h is from equation \ref{eq:EKF_2} and $ v_k $ is approximated to zero for finding an estimate of state. In the \textit{Kalman gain} and covariance equations, H and V are again Jacobian matrices of the partial of h with respect to x and v respectively. They are also recalculated each step using equations \ref{eq:EKF_10} and \ref{eq:EKF_11}

\begin{equation}
\label{eq:EKF_10}
H_{[i,j]} = \frac{\partial h_{[i]}}{\partial x_{[j]}} (\hat{x}_{k-1},u_k,0)
\end{equation}

\begin{equation}
\label{eq:EKF_11}
V_{[i,j]} = \frac{\partial h_{[i]}}{\partial v_{[j]}} (\hat{x}_{k-1},u_k,0)
\end{equation}

\section{Application of Extended Kalman Filter for SLAM}
\label{sec:EKF_SLAM}
In the case of Simultaneous Localization and Mapping, the state we are trying to estimate is a combination of the position of the robot as well as the environment. The variable $ x $ can be assumed as a vector of the robot's pose and the position of various features in the environment that we are interested in. These may be explicitly defined features or generic in nature. The error in our knowledge of the robot pose and our knowledge of the environment is represented by the covariance $ P $. 

Of the subparts mentioned in section \ref{sec:SLAM_parts}, it is now apparent that the \textit{State Estimation} step involves the \textit{Prediction} step of the \ekf. The function $ f $ from equation \ref{eq:EKF_1} is our system model. It typically involves the motion model of the robot. It is a function that gives us an estimate of the it's pose. As for the input $ u_k $, it can take any information about the motion that it undergoes in time step $ k $. It can be the commands given to the robot or , more commonly, the measurements of the proprioceptive sensors such as encoders. This motion model is differentiated with respect to both itself and the noise model to get the Jacobian matrices used in equation \ref{eq:EKF_4}. The specific motion and noise models used will be discussed in further chapters.

Now that we have an estimate of the robot pose at time $ k $, we need to improve upon it. For this we need a measurement of the environment using any exteroceptive sensor on the robot. This measurement is got in the \textit{Landmark Extraction} step. A wide variety of sensors can be used such as a laser range finders and cameras depending on both the environment and the robot. Whatever sensor is used, the feature or landmark which is to be used as measurement is extracted and it's relation to the pose of the robot is represented by a measurement model. This model forms the function $ h $ from equation \ref{eq:EKF_2}. Differentiating this with respect to the robot pose we get a part of the $ H $ matrix. 

For the rest of the Jacobian we need to differentiate it with respect to the rest of the state vector which contains some representation of the environment already observed by the robot. Also to successively calculate the difference $ z_k-h(\hat{x}^-_k,0) $ from equation \ref{eq:EKF_8}, commonly called the \textit{innovation}, we need to know which measurement corresponds to which feature in the observed environment. This is the essence of the \textit{Data Association} step. A number of methods such as Euclidean or Mahalanobis\cite{Mahalanobis1936} distance are commonly used. 

Once we have the innovation, both the robot pose and the environment is updated simultaneously using equations \ref{eq:EKF_7} to \ref{eq:EKF_9}.
\chapter{Integrated Mobile Platform}
\label{cha:Platform}

\begin{figure}
\centering
\includegraphics[width=0.5\textwidth]{IMP_color}
\caption{The Integrated Mobile Platform}
\end{figure}

\section{Overview}
\begin{figure}
\centering
\includegraphics[width=0.8\textwidth]{blockDiag}
\caption{System diagram of the IMP}
\label{fig: blockDiag}
\end{figure}
The primary purpose of building the \imp or the IMP is to test the various algorithms that are used in SLAM. Since SLAM is mostly used for real time control, it is desirable to have platform with the capabilities to implement that. For this the following considerations act as guidelines.
\begin{description}
	\item[Movement] Has to be able to move with a variety of speeds and relatively small turning radii as it is to be used indoors. 
	\item[Self-position acknowledgment] Has to have proprioceptive sensors which give an estimate of it's own position and orientation.
	\item[Environment sensing] Has to have sensors to understand the environment. 
	\item[On-board computer] Has to have sufficient on-board processing power to do the computations necessary for SLAM.
	\item[Real time controller] Has to have a capability to implement real time control.
	\item[Communication] Has to have robust communication between the different components and also with the ground station. 
	\item[Memory] Has to have sufficient on-board memory to collect data to enable testing of SLAM algorithms off line.  
	\item[Flexibility] The various components both hardware and software need to be designed in such a way that it is easy to switch them around.
\end{description}

In each stage these requirements act as guiding principles with emphasis given to the flexibility aspect as it is essentially a research platform.  
\section{Mechanical Design}

The IMP is designed as a six wheel differential drive platform with dimensions as seen in figure \ref{fig:IMPviews}. The four outer wheels are the driving wheels equipped with a motor each and the two inner ones are equipped with encoders as seen in figure \ref{fig:IMPback}. The motor and encoder specification are given in table \ref{tab: Components}. The two motors on the same side if IMP are connected together so that the same voltage can be given to both. The wheels are made of rubber with a offset-V tread to give good gripping on both indoor and outdoor environments. They are attached by a direct shaft going through two ball bearings. The rest of the controls, and sensors are mounted on railings on the topside of the platform. 

\begin{figure}
    \centering
    \begin{subfigure}[b]{0.3\textwidth}
	    \includegraphics[width=\textwidth]{IMP}
	    \caption{Isometric view}
	    \label{fig:IMP}
    \end{subfigure}
    \quad %add desired spacing between images, e. g. ~, \quad, \qquad, \hfill etc.
      %(or a blank line to force the subfigure onto a new line)
    \begin{subfigure}[b]{0.3\textwidth}
        \includegraphics[width=\textwidth]{IMPtop}
        \caption{Top view}
        \label{fig:IMPtop}
    \end{subfigure}%
    \quad %add desired spacing between images, e. g. ~, \quad, \qquad, \hfill etc.
      %(or a blank line to force the subfigure onto a new line)
    \begin{subfigure}[b]{0.3\textwidth}
        \includegraphics[width=\textwidth]{IMPside}
        \caption{Side view}
        \label{fig:IMPside}
    \end{subfigure}%
    \caption{Dimensions of \imp}
    \label{fig:IMPviews}
\end{figure}

\section{Electrical Components}
The major Electrical Components on the IMP can be classified under 4 main categories as shown in figure \ref{fig: blockDiag}. Starting with the power block, a Lithium Polymer battery supplying 12V is used and it is regulated on the Power regulation board,component J, to 5V for the Odroid which serves as the on-board computer(Component I on the table \ref{tab: Components}). As an autopilot board is used, the Inertial Measurement Unit and the micro-controller are integrated in the same component A in figure \ref{fig:IMPfront}. The other sensors are distributed between the autopilot and the Odroid. The motor driver of choice is the Sabertooth 2x12 as it has a differential command mode in which velocity and omega PWMs are converted to individual motor voltages in hardware. 

\begin{figure}[h!]
    \centering
    \begin{subfigure}[b]{0.4\textwidth}
	    \includegraphics[width=\textwidth]{IMPfront}
	    \caption{Top view}
	    \label{fig:IMPfront}
    \end{subfigure}
    \quad %add desired spacing between images, e. g. ~, \quad, \qquad, \hfill etc.
      %(or a blank line to force the subfigure onto a new line)
    \begin{subfigure}[b]{0.3\textwidth}
		\includegraphics[width=\textwidth]{IMPback}
		\caption{Bottom view}
		\label{fig:IMPback}
    \end{subfigure}%
    \caption{Electrical components on the \imp}
    \label{fig:IMPparts}
\end{figure}

%\begin{table}[h!]
\begin{longtable}{| l | p{3cm} | p{10cm} |}
		\caption{List of major components on the IMP}
		\label{tab: Components}\\

 		\hline
 		Label & Component & Description \\ \hline 
 		\endhead
 		
 		\hline \multicolumn{3}{|r|}{{Continued on next page}} \\ \hline
 		\endfoot
 		
 		\hline \hline
 		\endlastfoot
 		
 		A & Autopilot & A multi processor control and sensing unit. 
	 		\\ & & $ \bullet $ 220 Mhz RM48L952 primary micro-controller with 256 kB internal RAM and 3MB internal Flash 
	 		\\ & & $ \bullet $ 80 Mhz TM4C123GH6PZ secondary micro-controller with 32 kB internal RAM and 256 kB internal Flash 
	 		\\ & & $ \bullet $ 10 UARTs with 2 configurable for interprocessor communication 
	 		\\ & & $ \bullet $ 12 ADC inputs 
	 		\\ & & $ \bullet $ 16 PWM outputs	
 		\\ \hline
		B & ODROID-XU3 & The on-board computer
			\\ & & $ \bullet $ Samsung Exynos5422 Cortex™-A15 2.0Ghz quad core and Cortex™-A7 quad core CPUs
	 		\\ & & $ \bullet $ 2Gbyte LPDDR3 RAM at 933MHz (14.9GB/s memory bandwidth)
	 		\\ & & $ \bullet $ USB 3.0 Host x 1, USB 3.0 OTG x 1, USB 2.0 Host x4  
 		\\ \hline
 		C & Battery & 12V Lithium Polymer battery in 3S1P configuration with 2500 mAh capacity.
 		\\ \hline
 		D & Power Board & Provide multiple connections for 12 V power along with a battery monitoring circuit. Also provide 5V regulated power for the Odroid. 
 		\\ \hline
 		E & Motor Driver & Sabertooth 2x12 regenerative dual motor driver
			\\ & & $ \bullet $ 12A continuous, 25A peak per channel
			Up to 24V in
	 		\\ & & $ \bullet $ Synchronous regenerative drive
	 		\\ & & $ \bullet $ Ultra-sonic switching frequency
	 		\\ & & $ \bullet $ Input modes: Analog, R/C, simplified serial, packetized serial
 		\\ \hline
 		F & Motors & 165 RPM HD Precision Planetary Gear Motor
 			\\ & & $ \bullet $ Rated Voltage: 12VDC
 			\\ & & $ \bullet $ Rated Load: 7.3 kgf-cm (101.4 oz-in)
 			\\ & & $ \bullet $ Max. Stall Current: 20A @ 12VDC
 		\\ \hline
 		G & Encoders & MA3 Miniature Absolute Magnetic Shaft Encoder
 			\\ & & $ \bullet $ 10-bit Analog output - 2.6 kHz sampling rate
 			\\ & & $ \bullet $ Reports the shaft position over $ 360^\circ $ with no stops or gaps
 		\\ \hline
 		H & Camera & 5M HD USB Camera
 			\\ & & $ \bullet $ USB 2.0 compliance
  			\\ & & $ \bullet $ Focusing Range: 60cm to infinity	
   			\\ & & $ \bullet $ Fov (H): 72°
 		\\ \hline
 		I & LIDAR &  Hokuyo URG-04LX-UG01 Scanning Laser Rangefinder
 			\\ & & $ \bullet $ Detectable range of 20mm to 5600mm
 			\\ & & $ \bullet $ 100msec/scan
 			\\ & & $ \bullet $ 240° area scanning range with 0.36° angular resolution
 		\\ \hline
 		J & Wifi Module & TP-LINK Nano Wireless Travel Router
 			\\ & & $ \bullet $ 150Mbps Wi-Fi speed
 			\\ & & $ \bullet $ Supports AP, Client, Router, Repeater, and Bridge modes
 			\\ & & $ \bullet $ Compatible with 802.11b/g/n and 2.4GHz Wi-Fi devices
 			 			 			 		
 		\\ \hline 
	\end{longtable}
%\end{table}

\section{Control Process}
\begin{figure}[h]
\centering
\includegraphics[width=\textwidth]{commdiag}
\caption{Communication Diagram}
\label{fig: CommDiag}
\end{figure}
Another major part of the electrical circuits is the communication and information flow between the various components. The control decisions taken are mainly based on this. Each of the components have different means and protocols of communication. The communication setup is shown in figure \ref{fig: CommDiag} Once all the communication channels are established the information flow is designed as follows.
\begin{itemize}
	\item The Autopilot maintains timebase for the whole process. It also reads in the Encoder and IMU. All this information is sent to on-board computer through the serial port. It also logs all the data on an on-board SD card.
	\item The Odroid then reads the LIDAR and Camera data at that time. It logs both the data read in and the data it received from the autopilot. 
	\item The user commands are sent to the Odroid through the Wi-Fi and are transmitted as is or after interpretation as per the algorithm and the kind of commands being sent. 
\end{itemize}

\section{Encoder Based Odometry for State Estimate}

The encoders attached on the two passive wheels give a good estimate of the robot's movement in each time step. The function used to calculate the robot's movement from the encoder data is the motion model of the robot and is used for equation \ref{eq:EKF_1} in section \ref{sec:EKF}. Since this robot has only 3 degrees of freedom, the state vector, $ x \in \Re^3 $ and is defined as $ x = [x,y,\theta]^T $ where $ x $ and $ y $ give the position of the robot from an arbitrary fixed point in inertial frame of reference. 

\section{Motion model and the corresponding differentials}

To estimate the motion model, first left and right wheel movement is converted into distances traveled by them through a linear mapping using the known radius of the wheels. These values are represented by $ u = [ l, r ]^T $. The equations are better implemented as a piecewise function. The two cases of when the robot is estimated to be going straight or to be turning is considered separately. This decision is made by observing the distance traveled by the left and right wheels. If they are exactly the same, the robot is going straight and the motion model is given by equations \ref{eq:Enc_1}. 

If $ r = l $:
\begin{equation}
\label{eq:Enc_1}
	\begin{bmatrix}
		\hat{x}^-\\\hat{y}^-\\\hat{\theta}^-
	\end{bmatrix}_k
	=
	\begin{bmatrix}
		\hat{x}\\\hat{y}\\\hat{\theta}
	\end{bmatrix}_{k-1}
	+
	\begin{bmatrix}
		l.\cos(\hat{\theta}_{k-1})\\
		l.\sin(\hat{\theta}_{k-1})\\
		0
	\end{bmatrix}
\end{equation}	

If not,equations \ref{eq:Enc_2} and \ref{eq:Enc_3} are used. Where $ R,\alpha $ and $ w $ are as shown in figure \ref{fig:Enc_1}
			
\begin{figure}
\centering
\includegraphics[width=0.3\textwidth,height=0.3\textheight]{differential_turning}
\caption{Motion model of Differential drive platform}
\label{fig:Enc_1}
\end{figure}
If $ r \neq l $:
\begin{equation}
\label{eq:Enc_2}
	\alpha= \frac{r-l}{w}
\qquad
	R=\frac{l}{\alpha}
\end{equation}
\begin{equation}
\label{eq:Enc_3}
	\begin{bmatrix}
		\hat{x}^-\\\hat{y}^-\\\hat{\theta}^-
	\end{bmatrix}_k
	=
	\begin{bmatrix}
		\hat{x}\\\hat{y}\\\hat{\theta}
	\end{bmatrix}_{k-1}
	+
	\begin{bmatrix}
		\left(R+\frac{w}{2}\right)(\sin(\hat{\theta}_{k-1}+\alpha)-\sin(\hat{\theta}_{k-1}))\\
		\left(R+\frac{w}{2}\right)(-\cos(\hat{\theta}_{k-1}+\alpha)-\cos(\hat{\theta}_{k-1}))\\
		\alpha
	\end{bmatrix}
\end{equation}

In general for ease of representation this motion model is referred to only by $ f $. 
\begin{equation}[h!]
\label{eq:Enc_4}
\hat{x}^-_k = f(\hat{x}_{k-1},u_k)
\end{equation}

Once the motion model is known, it is necessary to find it's Jacobian with respect to the state. Since both the motion model $ f $ and the state have 3 dimensions the Jacobian will be a $ 3\times 3 $ matrix given by \ref{eq:Enc_5}. 

\begin{equation}
\label{eq:Enc_5}
A = \frac{\partial f}{\partial x} = 
\begin{bmatrix}
\frac{\partial f_1}{\partial x} & \frac{\partial f_1}{\partial y} & \frac{\partial f_1}{\partial z} \\
\frac{\partial f_2}{\partial x} & \frac{\partial f_2}{\partial y} & \frac{\partial f_2}{\partial z} \\
\frac{\partial f_3}{\partial x} & \frac{\partial f_3}{\partial y} & \frac{\partial f_3}{\partial z}
\end{bmatrix}
\end{equation}

Since the motion model is piecewise, it's Jacobian also calculated in two parts by differentiating the respective equations \ref{eq:Enc_1} and \ref{eq:Enc_3}.

If $ r = l $:
\begin{equation}
\label{eq:Enc_6}
A = 
\begin{bmatrix}
1 & 0 & -l\sin\theta\\
0 & 1 & -l\cos\theta\\
0 & 0 & 1
\end{bmatrix}
\end{equation}

If $ r \neq l $
\begin{equation}
\label{eq:Enc_7}
A = 
\begin{bmatrix}
1 & 0 & (R+\frac{w}{2})(\cos(\theta+\alpha)-\cos\theta)\\
0 & 1 & (R+\frac{w}{2})(\sin(\theta+\alpha)-\sin\theta)\\
0 & 0 & 1
\end{bmatrix}
\end{equation}

Where $ R,w $ and $ \alpha $ are according to equation \ref{eq:Enc_2} and figure \ref{fig:Enc_1}

Next the motion model is differentiated with respect to the noise. Here the process noise is assumed to be essentially due to the noise in encoder measurement and it is additive in nature. Hence it is possible to know the variance of the movement with respect to noise by differentiating the motion model with respect to the encoder measurements $ l $ and $ r $. Since this is of dimension 2, the noise covariance matrix W will be of dimension $ 3\times 2 $ given by equation \ref{eq:Enc_8}.

\begin{equation}
\label{eq:Enc_8}
W = \frac{\partial f}{\partial (u+w)} = \frac{\partial f}{\partial (u)} =
\begin{bmatrix}
\frac{\partial f_1}{\partial l} & \frac{\partial f_1}{\partial r}\\
\frac{\partial f_2}{\partial l} & \frac{\partial f_2}{\partial r}\\
\frac{\partial f_3}{\partial l} & \frac{\partial f_3}{\partial r}
\end{bmatrix} 
\end{equation}
Each of the individual terms are then calculated independently.
\begin{subequations}
If $ r=l $:
	\begin{align}
		\frac{\partial f_1}{\partial l} &= \frac{1}{2}(\cos\theta+\frac{l}{w}\sin\theta)\\
		\frac{\partial f_2}{\partial l} &= \frac{1}{2}(\sin\theta-\frac{l}{w}\cos\theta)\\
		\frac{\partial f_1}{\partial r} &= \frac{1}{2}(-\frac{l}{w}\sin\theta+\cos\theta)\\
		\frac{\partial f_2}{\partial r} &= \frac{1}{2}(\frac{l}{w}\cos\theta+\sin\theta)\\
		\frac{\partial f_3}{\partial l} &= -\frac{1}{w} \quad \frac{\partial f_3}{\partial r} = \frac{1}{w}
	\end{align}
\end{subequations}

\begin{subequations}
If $ r\neq l $:
	\begin{align}
		\frac{\partial f_1}{\partial l} &= \frac{wr}{(r-l)^2}(\sin\theta'-\sin\theta)-\frac{r+l}{2(r-l)}\cos\theta'\\
		\frac{\partial f_2}{\partial l} &= \frac{wr}{(r-l)^2}(-\cos\theta'+\cos\theta)-\frac{r+l}{2(r-l)}\sin\theta'\\
		\frac{\partial f_1}{\partial r} &= \frac{-wr}{(r-l)^2}(\sin\theta'-\sin\theta)+\frac{r+l}{2(r-l)}\cos\theta'\\
		\frac{\partial f_2}{\partial r} &= \frac{wr}{(r-l)^2}(-\cos\theta'+\cos\theta)-\frac{r+l}{2(r-l)}\sin\theta'\\
		\frac{\partial f_3}{\partial l} &= -\frac{1}{w} \quad \frac{\partial f_3}{\partial r} = \frac{1}{w}
	\end{align}
\end{subequations}

Where, $ \theta'=\theta+\alpha $ and $ R,w $ and $ \alpha $ are as per figure \ref{fig:Enc_1}. Once the Jacobian is calculated, the last component needed for the estimation according to equation \ref{eq:EKF_4} is the Process noise covariance $ Q $. This has to contain some information about the amount the noise in each time step. Since all noise is assumed to be only sensor measurement noise, a diagonal matrix with the error in each encoder is chosen as the covariance as in equation \ref{eq:Enc_9}.

\begin{equation}
\label{eq:Enc_9}
Q = 
\begin{bmatrix}
\sigma_l^2 & 0\\
0 & \sigma_r^2
\end{bmatrix}
\end{equation}

\subsection{Limitations of odometry based estimation}

One of the primary limitation of estimating a robot's position solely on odometry is that, over a long time, the error accumulates and result in the estimate being very far from the actual position of the robot. Other than this primary drawback, there are additional drawbacks. Every real world encoder has a component of noise which accumulates over time. Also there is a good chance of wheel slippage especially during turns making it hard to accurately reconstruct a turn. Another drawback is that if the encoder wheels are not fully lubricated and free to move, any wheel might stop moving with the robot and skid instead. This will result in a turn like reconstruction.

 
\section{Experimental results}
\textit{Description of the arena and the run performed.}

\textit{Images of the path ground truth and prediction.}

We see that while it gives a good estimate of the path taken, it gradually deviates from the actual truth.
\chapter{Feature Extraction with Point Features}

\section{Overview}

To improve the estimate of the state and to get an idea of our environment, we need to understand the input of our exteroceptive sensors. As seen in chapter \ref{cha:Platform}, our robot is equipped with both a Camera and a LIDAR. In this chapter we try to use a simple algorithm based to detect fixed features in the environment. We then try to improve upon the algorithm by using our preexisting knowledge of the environment.  

\section{Feature Extraction Algorithm}

Consider an extremely clean environment. One which has only fixed features and they are all away from the wall. And the entire environment is within the range of the LIDAR. In such an environment the distance readings gradually increase and decrease all along the walls except when they encounter and obstacle or a \textit{feature}. This algorithm essentially looks for large jumps in the distance readings. One such arena is shown in figure \ref{fig:Simulated_1}. The gradual nature of the distance readings are more clearly seen when we plot the scan values as a function of the angle of the scan beam with respect to the robot as done in figure \ref{fig:Vrep_plot}.
\begin{figure}[h!]
    \centering
    \begin{subfigure}[b]{0.3\textwidth}
	    \includegraphics[width=\textwidth]{3d_vrep}
	    \caption{A simulated arena}
	    \label{fig:3d_vrep}
    \end{subfigure}
    \quad %add desired spacing between images, e. g. ~, \quad, \qquad, \hfill etc.
      %(or a blank line to force the subfigure onto a new line)
    \begin{subfigure}[b]{0.3\textwidth}
        \includegraphics[width=\textwidth]{arena_vrep}
        \caption{Top view}
        \label{fig:arena_vrep}
    \end{subfigure}%
        \caption{Simulated Arena for LIDAR}
        \label{fig:Simulated_1}
\end{figure}

For the robot to detect the large jumps we differentiate the distance with respect to angles. As seen in figure \ref{fig:Vrep_cylinders} this will give a specific pattern each time a cylinder is present. A condition can then be designed in the following way. Each time the derivative is larger than a fixed threshold and is a negative number, a cylinder's beginning is found, and when a large positive value is encountered, it's end. Finding the mid point of these two readings we find the location of the cylinder as in figure \ref{fig:Vrep_cylinders}. 
\begin{figure}
        \centering

        \begin{subfigure}[b]{0.48\textwidth}
                \includegraphics[width=\textwidth]{Vrep_plot}
                \caption{LIDAR scan}
                \label{fig:Vrep_plot}
        \end{subfigure}
        \quad %add desired spacing between images, e. g. ~, \quad, \qquad, \hfill etc.
          %(or a blank line to force the subfigure onto a new line)
%        \begin{subfigure}[b]{0.3\textwidth}
%                \includegraphics[width=\textwidth]{Vrep_derivative}
%                \caption{Derivative of scan}
%                \label{fig:Vrep_derivative}
%        \end{subfigure}%
        \quad %add desired spacing between images, e. g. ~, \quad, \qquad, \hfill etc.
          %(or a blank line to force the subfigure onto a new line)
        \begin{subfigure}[b]{0.48\textwidth}
                \includegraphics[width=\textwidth]{Vrep_cylinders}
                \caption{Detected cylinders}
                \label{fig:Vrep_cylinders}
        \end{subfigure}

        \caption{Differential based LIDAR feature detection}\label{fig:Simulated_2}
\end{figure}

It is apparent this method has a large number of drawbacks. The arena has to be fully within the range of the sensor, if not there will be breaks in the distance curve that will generate spurious landmarks. This drawback can be easily overcome by creating a zero order hold for those regions. Also any disturbances in the boundaries will also generate spurious landmarks. The threshold for the derivative is a very sensitive tuning parameter. If we set it too high there is an increased chance of missing landmarks placed closed to walls and if it is set low, there are a large number of spurious landmarks. 

To overcome this, based on preexisting knowledge of the nature of features a filter of sorts can be designed. A low threshold is chosen and a large number of prospective features are enumerated. All those that lack a minimum number of points on them are immediately eliminated. Next, based on the fact that we know the radius of the cylinders that are the landmarks, we can approximate the angular width of any feature detected. From the large set, only the ones within a range around this approximate angular width are retained. Then, by computing least squares fit we see if the features are indeed cylinders or segments of the wall. Only the features having a large enough curvature are retained. By this method we add a small amount of robustness to a simplistic algorithm of feature detection. The results of such a filter when applied with EKF can be seen in section \ref{sec:Spike_results}.

Once the features are detected we need to determine the measurement model from the robot to these features to be used in \ekf SLAM. 

\section{Measurement Model and the corresponding Jacobian matrices}
\label{sec:Spike_math}

Once we have found the cylinder coordinates in the LIDAR data, we need to store it's position in the map. Since we know the estimate of the robot's position in the inertial frame, we can map the cylinder coordinates from robot to inertial frame using a homogeneous transformation as in equation \ref{eq:SpikeMath1}. Where $ (x_w,y_w) $ are coordinates in world frame. $ (x_r,y_r,\theta_r) $ are estimated pose of the robot and $ (x_1,y_1) $
\begin{equation}
\begin{bmatrix}
x_w\\y_w\\1
\end{bmatrix}=
\begin{bmatrix}
\cos(\theta_r) & -\sin(\theta_r) & x_r\\
\cos(\theta_r) & \sin(\theta_r) & y_r\\
0 & 0 & 1
\end{bmatrix}
\begin{bmatrix}
x_1\\y_1\\1
\end{bmatrix}
\label{eq:SpikeMath1}
\end{equation}

Once we have the cylinder we need to check if it corresponds to any preexisting cylinder in the map. For this the euclidean distance between the detected cylinder and the existing cylinder positions is measured and if it is less than a particular threshold then the new cylinder is associated with the existing one. 

For the correction of the robot pose as explained in section \ref{sec:EKF_SLAM}, we need to find the measurement model of the point features. That is we need the laser reading or the distance $ r $ and angle $ \alpha $ from the robot to the cylinder. For this we first find the coordinates of the cylinder in the robots frame of reference using the inverse of the homogeneous transformation that we use in equation \ref{eq:SpikeMath1}. But the robot pose used now is the new position at this particular time step. Once we have $ (x_1,y_1) $ in the robot frame we can find the measurement h using equation \ref{eq:SpikeMath2}. Same equations with the detected cylinder coordinates will give $ z $ allowing us to calculate the \textit{innovation} for use in equation \ref{eq:EKF_8}. 

\begin{equation}
	\label{eq:SpikeMath2}
	h=\begin{bmatrix}
	r\\\alpha
	\end{bmatrix}=
	\begin{bmatrix}
	\sqrt{(y_1-y_r)^2+(x_1-x_r)^2} \\
	\tan^{-1}\left(\frac{y_1-y_r}{x_1-x_r}\right)-\theta_r
	\end{bmatrix}
\end{equation}

Once we have the measurement model to find the Kalman Gain according to equation \ref{eq:EKF_7} we need the derivatives of it with respect to the state vector $ x \in \Re^n $ which contains the robot pose as well as all the landmarks already existing in the map at that particular time step. Since we assume each landmark is independent of each other. most part of the Jacobian $ H $ contains zeros except for the part corresponding to the robot pose and if the measurement has been associated with an existing landmark, then it will depend on that landmark's position. Hence for the Jacobian $ H $ we differentiate $ h $ as per equation \ref{eq:SpikeMath3}. 

\begin{equation}
\label{eq:SpikeMath3}
	H = 
	\begin{bmatrix}
	\frac{\partial r}{\partial x_r} & \frac{\partial r}{\partial y_r} & \frac{\partial r}{\partial \theta_r} & \cdots & \frac{\partial r}{\partial x_1} & \frac{\partial r}{\partial y_1} & \cdots \\
	\frac{\partial \alpha}{\partial x_r} & \frac{\partial \alpha}{\partial y_r} & \frac{\partial \alpha}{\partial \theta_r} & \cdots & \frac{\partial \alpha}{\partial x_1} & \frac{\partial \alpha}{\partial y_1} & \cdots 
	\end{bmatrix}
\end{equation}

We derive each of the terms separately as per equation \ref{eq:SpikeMath4}. 

\begin{subequations}
\label{eq:SpikeMath4}
	\begin{align}
	\frac{\partial r}{\partial x_r} &= \frac{(x_r-x_1)}{\sqrt{(y_1-y_r)^2+(x_1-x_r)^2}} \qquad
	\frac{\partial \alpha}{\partial x_r} =  
	\frac{(y_1-y_r)}{(y_1-y_r)^2+(x_1-x_r)^2} \\
	\frac{\partial r}{\partial y_r} &= \frac{(y_r-y_1)}{\sqrt{(y_1-y_r)^2+(x_1-x_r)^2}} \qquad
	\frac{\partial \alpha}{\partial y_r} =  
	\frac{(y_r-y_1)}{(y_1-y_r)^2+(x_1-x_r)^2} \\
	\frac{\partial r}{\partial \theta_r} &= 0 \qquad  
	\frac{\partial \alpha}{\partial \theta_r} = -1\\
	\frac{\partial r}{\partial x_1} &= \frac{(x_r-x_1)}{\sqrt{(y_1-y_r)^2+(x_1-x_r)^2}} \qquad
	\frac{\partial \alpha}{\partial y_1} =  
	\frac{(y_1-y_r)}{(y_1-y_r)^2+(x_1-x_r)^2} \\
	\frac{\partial r}{\partial x_1} &= \frac{(y_r-y_1)}{\sqrt{(y_1-y_r)^2+(x_1-x_r)^2}} \qquad
	\frac{\partial \alpha}{\partial y_1} =  
	\frac{(y_r-y_1)}{(y_1-y_r)^2+(x_1-x_r)^2}	
	\end{align}
\end{subequations}

The only other information we need for Kalman Gain calculation is the observer error $ V^TRV $ with V being the derivative of the measurement model with respect to noise. If we assume all landmarks are uniformly affected by noise, we can assume V to be identity. R is the measurement noise. It is a diagonal matrix with one of the eigenvalues representing the error in distance measurement of the LIDAR and the other the angle measurement as in equation \ref{eq:SpikeMath5}. 
\begin{equation}
\label{eq:SpikeMath5}
R = 
\begin{bmatrix}
\sigma_r & 0 \\
0 & \sigma_\alpha
\end{bmatrix}
\end{equation}

Using all this information we can correct our estimate of the robot pose while simultaneously correcting the position of the landmarks. 
\section{Experimental results}
\label{sec:Spike_results}
\textit{Description of the arena and the run performed.}

\textit{Images of the path ground truth and correction.}

We see that it estimates the path taken by the robot to a good extant as well as gives us an idea of the environment.
 
\section{Feature Extraction with Linear Features}
\label{sec:extraction}
\subsection{Overview}
One of the most ubiquitous features of any indoor environment are walls. While it may be necessary to find other obstacles and landmarks for path planning and mapping, knowing the locations of walls tends to be advantageous. At the very least, it can be used to remove all the laser readings that are close to a wall, therefore, reducing the data set to be further analyzed by methods such as clustering. In relatively clean environments such as corridors that are not too crowded, walls will usually suffice for \slam. Also, algorithms that find linear features are more robust to humans moving around in the environment than those finding cylinders or generic features in the data. Also since the feature of interest is a wall, it is reasonable to model it as a line of infinite length which reduces the dimensions of state required to represent it. Similar to Section~\ref{sec:Spike_math}, once the walls have been detected, a measurement model is required to calculate the innovation. It is obvious that the measurement model and its derivatives will not be the same as in point features, as the apparent relative motion of a wall as the robot moves is very different than for point features. For example, when the robot is moving parallel to the wall, the LIDAR scans from one time step to another will be identical, but unlike in the case of point features the similarity of LIDAR scans over successive time steps does not imply that the robot has not moved.

\subsection{RANSAC Based Feature Extraction Algorithm}
\label{sec:ransac}
The fundamental concept in extracting linear features is to fit a line or multiple lines onto a set of points. There are a large number of methods to do this such as least squares, Split-merge and RANSAC\cite{Nguyen2005}. The typical RANSAC-based line extraction in libraries such as OPENCV and PCL, attempt to robustly fit a line to a given set of points. In typical environments, since more than one wall can be seen at a time, it becomes necessary to first segment the data before using the algorithm for demo. Instead, another approach is to randomly sample points and try to fit multiple lines simultaneously.RANSAC finds these line landmarks by randomly taking a sample of the laser readings and then using a least squares approximation to find the best fit line that runs  through these readings. Once this is done RANSAC checks how many laser readings  lie close to this best fit line. If the number is above some threshold we can safely  assume that we have seen a line (and thus seen a wall segment). This threshold is  called the consensus. This approach is explained in algorithm~\ref{alg: RANSAC algorithm}\cite{riisgaard2003slam}. 

\begin{algorithm}[H]
\begin{algorithmic}

	\While{
	\begin{itemize}
		\item there are still unassociated laser readings, 
		\item and the number of readings is larger than the consensus, 
		\item and less than N trials have been done,
	\end{itemize}
	}
	\begin{itemize}
	\item Select a random laser data reading. 
	\item Randomly sample S data readings within D degrees of this laser 
	data reading 
	\item Using these S samples and the original reading calculate a 
	least squares best fit line $ L $. 
	\item Determine how many laser data readings lie within X distance 
	from this best fit line
	\item If the number of laser data readings on the line is above some 
	consensus C do the following: 
		\begin{itemize}
			\item calculate new least squares best fit line based on all 
			the laser readings determined to lie close to the old best fit 
			line. 
			\item Add this best fit line to the lines that have been extracted. 
			\item Remove the number of readings lying close to the line from the 
			total set of unassociated readings.
		\end{itemize}
	\end{itemize}
	\EndWhile
\end{algorithmic}
	\caption{Multiple line fitting with RANSAC.}
	\label{alg: RANSAC algorithm}
\end{algorithm}
This algorithm can thus be tuned based on the following parameters: 
\begin{itemize}
\item N – Max number of times to attempt to find lines. 
\item S – Number of samples to compute initial line. 
\item D – Degrees from initial reading to sample from. 
\item X – Max distance a reading may be from line to get associated to line. 
\item C – Number of points that must lie on a line for itto be taken as a line
\end{itemize}


\subsection{RANSAC Examples}
 \begin{figure}[h!]
     \centering
     \begin{subfigure}[b]{0.45\textwidth}
     
 	    \includegraphics[width=\textwidth]{blank_scan}
         \caption{Polar plot of laser scan.}
     \end{subfigure}
     \quad %add desired spacing between images, e. g. ~, \quad, \qquad, \hfill etc.
       %(or a blank line to force the subFigure~onto a new line)
     \begin{subfigure}[b]{0.45\textwidth}
         \includegraphics[width=\textwidth]{ransac_good}
		 \caption{Detected lines.}
     \end{subfigure}%
         \caption{Example of walls detected by RANSAC.}
         \label{fig: ransac_good}
 \end{figure}
 Considering the same setup as in Figure~\ref{fig:Real world arena}, we can try the RANSAC algorithm on the data collected from the LIDAR. It is initially seen to give good results as seen in Figure~\ref{fig: ransac_good}. However, the robustness of the algorithm is not consistant. As we move forward though, the robustnes is not always maintained. Looking at algorithm~\ref{alg: RANSAC algorithm} it is possible that since the first laser reading is chosen at random, it may not be on a wall at all. S readings in its neighborhood could generate a line that is at a small angle to both walls in the environment. Since it is close enough to the walls, a large number of points will wrongly be associated to this, which will result in a spurious wall being detected.This will have a detrimental effect in all the further time steps. An example of this is seen in Figure~\ref{fig: ransac_bad}.
 \begin{figure}[h!]
     \centering
     \begin{subfigure}[b]{0.45\textwidth}
     
 	    \includegraphics[width=\textwidth]{ransac_bad}
        \caption{Detected lines.}
     \end{subfigure}
     \quad %add desired spacing between images, e. g. ~, \quad, \qquad, \hfill etc.
       %(or a blank line to force the subFigure~onto a new line)
     \begin{subfigure}[b]{0.45\textwidth}
         \includegraphics[width=\textwidth]{ransac_bad1}
		 \caption{Detected lines.}
     \end{subfigure}%
         \caption{Example of spurious walls marked by RANSAC.}
         \label{fig: ransac_bad}
 \end{figure}
\subsection{Hough Transform Based Feature Extraction Algorithm}
\label{sec:hough}
In the fields of image analysis and computer vision, the Hough transform is a very popular feature extraction technique\cite{Stockman2001}. While it was originally designed to find lines in an image, it can also be used to detect any shape as long as that shape can be represented in a parametric form\cite{Duda1972}. This algorithm can detect a shape in spite of distortions making it much more robust than the RANSAC based algorithm shown in Algorithm~\ref{alg: RANSAC algorithm}\cite{Hu1998}. In the current application, the aim is to find walls in the range data of the LIDAR. Hence, the first step is to rasterize the points into an image so that we can use the OpenCV implementation of Hough transform. The image is then converted to a binary image and passed to the Hough transform function. The algorithm itself is described below.

In image processing, a common way of representing lines is the normal form of a line equation, 
\begin{equation}
r=x\cos\theta+y\sin\theta.
\label{eq:hough1}
\end{equation}
In this form a line is completely defined by 2 variables $ r $ and  $ \theta $. Note that for any point $ (x_0,y_0) $ in the image, a family of lines passing through that point is expressed as,

\begin{equation}
r_\theta=x_0\cos\theta+y_0\sin\theta.
\label{eq:hough2}
\end{equation}

Since all lines are assumed to be infinite in both directions, $ \theta $ can only take unique values between $ 0^\circ $ and $ 180^\circ $. This range is then discretized to give a fixed number of possible line angles. The range $ r $ is also discretized based on the resolution required. For single pixel resolution, the possible ranges will be all numbers from 0 to the length of the diagonal of the image. Since this will result in a large number of possible $ r $ values, a lower resolution will considerably speed up the algorithm. Now that there is a fixed number for possible $ r $ and $ \theta $, a 2D matrix of zeros is created in which each row corresponds to a possible value of $ \theta $ and each column to $ r $ as in Equation~\ref{eq:hough3}. This is called the accumulator:

\begin{equation}
A=\bordermatrix{~  & r_1 & r_2&\ldots & r_n \cr
              \theta_1& 0 &  0  & \ldots & 0\cr
              \theta_2& 0  &  0 & \ldots & \cr
              \vdots& \vdots & \vdots & \ddots & \vdots\cr
              \theta_n& 0  &   0       &\ldots & 0}.
\label{eq:hough3}
\end{equation} 

 Once this initial setup is complete, a non zero point on the binary image $ (x_0,y_0) $ is chosen at random. For this point, the family of lines passing through it will be given by Equation~\ref{eq:hough2}. For each possible $ \theta $ the corresponding $ r $ is calculated from the same equation. For each $ (r,\theta) $ pair found, the corresponding entry in the accumulator $ A $ is incremented. Once all the possible pairs are calculated, a different image point is chosen till all the non zero points on the image are explored. 
 
 The accumulator now represents all possible lines in the image and the number of points that lie on each of the lines. The accumulator entries that are greater than a threshold represent the lines in the image. The $ (r,\theta) $ values corresponding to these lines are returned. 
 
 An important aspect of using the ransform from OpenCV is the conversion of LIDAR data to an image and the detected lines back to the robot frame of reference. The former is straightforward. Since the range of the LIDAR is known, the image size can be fixed based on the resolution required. For, example the LIDAR described in Chapter~\ref{cha:Platform } has a maximum range of 5 m. So for a resolution of 1 cm, a blank image of $ 500 \times 500 $ pixels is created. When each LIDAR point is converted to Cartesian coordinates $ (x_l,y_l) $, they are with respect to an origin at the LIDAR itself. OpenCV images on the other hand follow a coordinate system that has the origin in the top right corner of the image. Hence, each point is mapped to the image coordinate frame using a homogeneous transformation given as
 \begin{equation}
 \begin{bmatrix}
 x_i\\y_i\\1
 \end{bmatrix}=
 \begin{bmatrix}
 0 & -1 & 500\\
 1 & 0 & 500\\
 0 & 0 & 1
 \end{bmatrix}
 \begin{bmatrix}
 x_l\\y_l\\1
 \end{bmatrix}.
 \label{eq:hough4}
 \end{equation}
 
The image points are then rounded to the nearest centimeter and the pixel corresponding to that value is set to white. This gives an image with black background and all the Laser readings shown in white which can be passed to the Hough transform function.

% It may sometimes be required to erode and dilate the image with different kernels to make the points more visible. 
 
The lines returned by the Hough transform, are in units of pixels and are with reference the image origin. To convert it to the robot frame, 2 points are chosen on the line and each of these points are mapped to the robot frame using the inverse transformation of Equation~\ref{eq:hough4} and the normal form of the equation of a line passing through these 2 points is found. 

\subsection{Hough Transform Examples}
In the same arena as in Figure~\ref{fig:Real world arena}, the Hough transform is found to be really robust in finding the walls as seen in Figure~\ref{fig: hough1}.
 \begin{figure}[h!]
     \centering
     \begin{subfigure}[b]{0.45\textwidth}
     
 	    \includegraphics[width=\textwidth]{blank_scan}
         \caption{Polar plot of laser scan.}
     \end{subfigure}
     \quad %add desired spacing between images, e. g. ~, \quad, \qquad, \hfill etc.
       %(or a blank line to force the subFigure~onto a new line)
     \begin{subfigure}[b]{0.45\textwidth}
         \includegraphics[width=\textwidth]{hough1}
		 \caption{Detected lines.}
     \end{subfigure}%
         \caption{Example of walls detected by Hough transform.}
         \label{fig: hough1}
 \end{figure}

Unlike the RANSAC algorithm, the Hough transform considers all possible lines before deciding on any line and is robust to noise in the surroundings. This is especially useful when there are people walking around in the environment.
 \begin{figure}[h!]
     \centering
     \begin{subfigure}[b]{0.45\textwidth}
     
 	    \includegraphics[width=\textwidth]{overhead2}
         \caption{Overhead view of the arena.}
     \end{subfigure}
     \quad %add desired spacing between images, e. g. ~, \quad, \qquad, \hfill etc.
       %(or a blank line to force the subFigure~onto a new line)
     \begin{subfigure}[b]{0.45\textwidth}
         \includegraphics[width=\textwidth]{hough2}
		 \caption{Detected lines}
     \end{subfigure}%
         \caption{Example 1 of Hough transform being robust to additional objects (e.g.,\ people) in the environment.} 
         \label{fig: hough2}
 \end{figure}
 
  \begin{figure}[h!]
      \centering
      \begin{subfigure}[b]{0.45\textwidth}
      
  	    \includegraphics[width=\textwidth]{overhead3}
          \caption{Overhead view of the arena}
      \end{subfigure}
      \quad %add desired spacing between images, e. g. ~, \quad, \qquad, \hfill etc.
        %(or a blank line to force the subFigure~onto a new line)
      \begin{subfigure}[b]{0.45\textwidth}
          \includegraphics[width=\textwidth]{hough3}
 		 \caption{Detected lines}
      \end{subfigure}%
          \caption{Example 1 of Hough transform being robust to additional objects (e.g.,\ people) in the environment.}
          \label{fig: hough3}
  \end{figure}
 
\subsection{Measurement Model and the Corresponding Differentials}
\label{sec:linear_math}
Once the lines in the data are recovered, the lines need to be expressed in an effective format to store them. The common, slope-point form has a major disadvantage when trying to represent perfectly vertical lines and the two point form will expand the size of the measurement vector $ z $ increasing the calculation required for \ekf. So it is preferable to represent it in the normal form where just the coordinates of the point of intersection of the normal from the origin to the line is stored. For example, in Figure~\ref{fig: normal_form}, the line can be represented solely by the point $ D $.

\begin{figure}
\centering
\includegraphics[width=0.45\textwidth]{normal_form_2}
\caption{Normal form of a line.}
\label{fig: normal_form}
\end{figure}

As in Section~\ref{sec:Spike_math}, the line coordinates have to be converted to the inertial frame of reference.For Example, in Figure~\ref{fig: wall_to_world}, the point $ P_1 $ in robot frame is known and point $ P_2 $ has to be found in the inertial frame given coordinates of the robot in inertial frame as $ P_r $(refer Algorithm~\ref{alg: wall_to_world}).
\begin{figure}
\centering
\includegraphics[width=0.45\textwidth]{wall_to_world}
\caption{Line in robot frame and world frame.}
\label{fig: wall_to_world}
\end{figure}

\begin{algorithm}[]
	\begin{enumerate}
		\item Convert point $ P_1 $ from robot frame to inertial frame as in Section~\ref{sec:Spike_math}
		\item Given points $ P_r $ and $ P_1 $ in inertial frame, the equation of line $ l_1 $ can be foundby connecting few points. 
		\item Since $ l1 \perp l2 $, the slope of $ l_2 $ is the negative reciprocal of the slope of $ l_1 $. 
		\item Using the slope and the point $ P_1 $ the Equation~of line $ l_2 $ is found using slope-point form of a line.
		\item The equation~of line l3 is found similarly using slope of $ l_1 $ and point $ O $. 
		\item The coordinates of point $ P_2 $ are obtained by solving equations for $ l_3 $ and $ l_2 $.
	\end{enumerate}
\caption{To convert linear features from robot frame to inertial frame.}
\label{alg: wall_to_world}
\end{algorithm}

Once the landmark is in the world frame,its corresponds to any existing landmark in the map can be checked. This can be checked by comparing both the perpendicular distances between the walls and the angle between the walls. This allows us to have 2 tuning parameters so that the association can be weighted as desired. Usually, since most walls in an indoor environments are at right angles, the variation allowed in angle for the association is larger compared to distance. 

Algorithm~\ref{alg: wall_to_world} is also called the inverse measurement model since in subsequent time steps for the measurement model is obtained by the exact opposite process: That is given a line using its normal form in inertial frame, get the \textit{measurement} from the robot. That is, we need the perpendicular distance from the robot to the wall and the angle of the laser beam which would hit the wall perpendicularly. For this given coordinate pf point $ P_2 $ as $ (x_2,y_2) $ and the robot pose as $ (x_r,y_r,\theta_r) $, the Cartesian coordinates $ P_1 $ is calculated as: % using Equation~\ref{eq:lineModel1}. 

\begin{equation}
	\label{eq:lineModel1}
	x_1 = x_2 - \frac{y_2 ( x_2 y_ r- y_2 x_r)}{(x_2^2 + y_2^2)}
	\qquad
	y_1 = y_2 + \frac{x_2 ( x_2 y_ r- y_2 x_r)}{(x_2^2 + y_2^2)}.
\end{equation}

Once $ P_1 $ is known,the perpendicular distance $ r $ and angle $ \alpha $ are calculated using Equation~\ref{eq:lineModel2}:
\begin{equation}
	\label{eq:lineModel2}
	r=\sqrt{(y_1-y_r)^2+(x_1-x_r)^2}
	\qquad
	\alpha = \tan^{-1}\left(\frac{y_1-y_r}{x_1-x_r}\right)-\theta_r,
\end{equation}
giving measurement model $ h $ as per Equation~\ref{eq:lineModel3}
\begin{equation}
	\label{eq:lineModel3}
	h=\begin{bmatrix}
	r\\\alpha
	\end{bmatrix}.
\end{equation}

For calculating the \textit{Kalman gain}, we need the Jacobian matrix $ H $. As each wall is independent of the other, this has the same structure as explained in Section~\ref{sec:Spike_math} and is given by Equation~\ref{eq:SpikeMath3}. Each of the terms on H is derived individually. The 
 A factor common to all the components of the Jacobian is given by:

  \begin{equation}	
	\beta = \frac{x_rx_1-x_1^2-y_1^2+y_ry_1}{x_1^2+y_1^2}.
  \end{equation}

The part of the Jacobian with respect to the position of the robot is:
  \begin{equation}
	\frac{\partial r}{\partial x_r} = 2x_1\beta \quad
	\frac{\partial r}{\partial y_r} = 2y_1\beta \quad
	\frac{\partial r}{\partial \theta_r} = 0 
  \end{equation}
  \begin{equation}
	\frac{\partial \alpha}{\partial x_r} = 0 \quad 
	\frac{\partial \alpha}{\partial y_r} = 0 \quad 
	\frac{\partial \alpha}{\partial \theta_r} = -1.
  \end{equation}
In these equations, it is seen that, $ \alpha $ is independent of the location of the robot and is inversely proportional to the orientation. Since $ r $ and $ \alpha $ are in the robot frame of reference. For a given wall, as long as the robot is facing the same direction, the angle at which the wall is seen remains the same. Similarly the perpendicular distance $ r $ between a robot center and a wall will remain the same irrespective of the orientation of the robot. 

Differentiating with the obstacle positions gives us the following:
	\begin{equation}
  \frac{\partial r}{\partial x_1} = -2x_1\beta^2 - 2\beta(2x_1-x_r)
  \end{equation}

  \begin{equation}
	\frac{\partial r}{\partial y_1} = -2y_1\beta^2 - 2\beta(2y_1-y_r)
	\end{equation}
  \begin{equation}
  \frac{\partial \alpha}{\partial x_1} =  \frac{-y_1}{x_1^2+y_1^2}\qquad
	\frac{\partial \alpha}{\partial y_1} =
	\frac{x_1}{x_1^2+y_1^2} .
	\end{equation}


Having the Jacobian matrices we can use the same observer error given by Equation~\ref{eq:SpikeMath5} to correct the position estimate using equations~\ref{eq:EKF_7} to~\ref{eq:EKF_9}.





\section{Measurement Model and the corresponding differentials}
\textit{The mathematical model for measurement.}

\section{Experimental results}
\textit{Description of the arena and the run performed.}

\textit{Images of the path ground truth and correction.}

\textit{It gives good estimate of walls and is much faster.}
 
\chapter{Visual odometry}

\section{Overview}

Traditionally a Camera is used as and exteroceptive sensor to get measurements of the environment. But it is interesting how it can also be used as a proprioceptive sensor using visual odometry\cite{}. By this, the number of sensors on a robot can be reduced saving cost and weight which are of extreme importance in certain cases. 


\section{Conceptual Description}
\textit{Description of epipolar geometry}
\textit{The mathematical base for visual odometry}

\section{Implementation and integration with EKF SLAM}
\textit{Visual odometry based motion model and Jacobian}

\section{Experimental results}
\textit{Description of the arena and the run performed.}
\textit{Images of the path ground truth and prediction.}
\textit{Images of the path ground truth and correction.}

Not as good. But good enough. Shown using corrected path
 
%\chapter{Implementation and Results}
\label{cha:results}

\section{Implementation of EKF SLAM Algorithm}
\label{sec:slam_process}
The SLAM algorithm, as discussed in Chapter~\ref{cha:Overview}, consists of subparts each of which present challenges of its own. The primary purpose of this thesis is to develop a flexible implementation that can accommodate different algorithms for the subparts,hence, the guiding principle of the implementation is modularity. Hence an object oriented language such as Python is the platform of choice. This allows us to create individual objects for each part of the algorithm which can be substituted with ease to utilize different combinations of algorithms for feature extraction, association and filtering. 

During the initial setup, all the objects are created and initialized. A robot object is instantiated with its initial position and covariance representing the uncertainty in its position. This object holds the robot's position and uncertainty all through the runtime. It also contains objects for all the sensors and components that the actual robot contains. For example, an object of ENCODER class is instantiated with all its properties such as its noise factors, the diameter of the wheels it is attached on, and the separation between the wheels. A LIDAR object is also instantiated with the minimum and maximum range of the Hokuyo, the offset of the LIDAR from the center of the platform, and the measurement noise factors. These sensor objects are attached to the Robot object into proprioceptive and exteroceptive sensor lists, respectively. The proprioceptive sensors contain methods to calculate the motion model of the robot and also its Jacobian matrices.
To the exteroceptive sensors, feature extractor objects are attached. Each feature extraction algorithm is implemented as its own class. All such classes have to have a particular structure based on the sensor whose data the extractor uses. For LIDAR based extractors,  there has to be a function called \textit{get\_landmarks()} which takes in angles and distances and returns a list of landmark locations. They also need a member variable named \textit{landmarkType} which contains the type of landmark that the extractor is trying to find. 

The exteroceptive sensors are designed to take the list of landmark positions from each feature extractor attached to them and create objects of LANDMARK class for them. A combined list of landmark objects from all the feature extractors is returned by the exteroceptive sensor class. These landmark objects are objects of the LANDMARK class. This is a factory class which, for example, returns objects of either WALL or CYLINDER based on the \textit{landmarkType} returned by \textit{get\_landmarks}. Each of these classes contain methods to calculate their measurement models and their Jacobian matrices.
The main part of SLAM is carried out by an object of EKF class. This object contains  state and covariance members which represent the whole environment. In SLAM along with the robot's position being updated, the environment also needs to be mapped. So a representation of the environment called the map is held by an object of EKF class. It is essentially a list of all robot and landmark objects created.

\begin{figure}
\centering
\includegraphics[width=\textwidth]{slam_process}
\caption{SLAM Implementation using only LIDAR.}
\label{fig:slam_process}
\end{figure}
Once initial setup is made, the robot object is added to the EKF map. Then for each time step, the \textit{run()} method of the EKF object is called. In this function the first step is prediction wherein for each robot in the map, the \textit{predict()} method of the robot object is called. That in turn calls the \textit{update()} method of all the proprioceptive sensors attached to that robot. This gives the estimated position and uncertainty using Equation~\ref{eq:Enc_3}. 

Then the \textit{observe()} method of the robot is called which in turn calls the exteroceptive sensors which generate landmark objects for all the landmarks seen at that time step. Each of these landmark objects are then checked to see if they are associated with a landmark preexisting in the map. For each re-observed landmark, the innovation and Kalman gain are calculated and the whole state is corrected. The map objects are then updated with the corrected positions. 

\section{Experimental Results}

To test the impact of the two feature extraction algorithms discussed in Chapter~\ref{cha:featureExtractor} on \slam, the \imp was driven around in first a controlled and then an uncontrolled indoor environment while continuously logging data. The Encoder and IMU data was logged at high speed at 100 Hz, and the LIDAR and Camera was logged at 5 Hz. The logged data was then analyzed using an EKF SLAM algorithm implemented as described in Section~\ref{sec:slam_process}. 

In the controlled environment, the two ends of a regular corridor are sealed off resulting in a closed space with no other features except walls and cylinders placed intentionally. This enables us to test the point feature extraction algorithm and see its effect on SLAM both individually and in combination with Hough transform based linear feature extraction. The controlled environment along with the path taken by IMP is seen in Figure~\ref{fig:overhead_path}. The closed-off space is still large enough such that it is not entirely within the range of the LIDAR. Hence, the problems and modifications discussed in Chapter~\ref{cha:featureExtractor} are still relevant even in a clean environment. 
\begin{figure}
\centering
\includegraphics[height = 0.9\textheight]{overhead_path}
\caption{Controlled environment and path reconstructed by an overhead camera.}
\label{fig:overhead_path}
\end{figure}
To look at the effectiveness of the SLAM algorithm, some knowledge of the actual path taken by the robot is required for comparison. This is reconstructed using an overhead camera and the movement of the robot in the video is recorded. While this may not give the actual path or the ground truth accurately, it gives a reasonable estimate of it to within a few inches of deviation. A sample reconstructed path is shown in Figure~\ref{fig:overhead_path}. The camera based reconstructed path is also shown in all the plots of SLAM reconstructions for comparison. In the run, the robot starts at the bottom right corner and ends at the position shown.

Since the map is constructed relative to the starting location, the robot is always assumed to be starting at the origin in all the reconstructions.

\subsection{Using Point Features}
\label{sec: point_result}
In Figure~\ref{fig:overhead_path} the cylinders around which the robot travels are the point features that are to be detected. The algorithm used is the same as described in Section~\ref{sec: spikeAlgo}. Using just the jump in distance as a landmark results in a large number of spurious landmarks. As discussed in Section~\ref{sec: spikeAlgo}, it is therefore necessary to filter through the candidate landmarks using preexisting knowledge as previously described. The minimum number of points that need to be on the candidate landmark is chosen to be 11 for this particular arrangement. This is a tuning parameter which is decided after a few trials for optimized performance. The other tuning parameters include the thresholding for the range that the angular width can be in based on the radius of the cylinders which is $ 0.1524 $ m and the threshold for the curvature of the candidate landmark which is chosen to be $ 9 \% $. These tuning parameters are chosen only once for a particular environment. 

Once the feature extractor is tuned, the SLAM algorithm itself needs to be tuned based on the estimated noise factors for the various sensors. The SLAM algorithm is more robust to these parameters than the feature extractor and hence, a range of values yield good performance. Another part of the algorithm that needs to be tuned carefully is the landmark association. For point features in the map, the Euclidean distance between the detected and the preexisting cylinder is used. Hence, the maximum allowable distance for association needs to be set. That distance has to be large enough to accommodate for a particular landmark not being seen for a few steps and then being detected. It also should not be too large since this will result in wrong associations which will correct the path of the robot in a completely different way than the correct associations. Hence this parameter is also a function of the order of distances in the arena and is chosen to be $ 2 m $, in these experiments.
\begin{figure}
\centering
\includegraphics[height = 0.9\textheight]{cylinder}
\caption{Controlled environment and path reconstructed by detecting point features in LIDAR data. Path starts from the origin.}
\label{fig:cylinder_result}
\end{figure}

With these parameters, the path of the robot is reconstructed as seen in Figure~\ref{fig:cylinder_result}. We can see that it tries to correct the path towards the ground truth compared to the odometry based estimate, but it overcompensates frequently. This demonstrates the disadvantage of the conservative nature of the filters used. Since the focus is to not allow any spurious landmarks, the actual landmarks are also missed in a number of time steps. This effects the reconstruction mainly when there is only one landmark in the field of view of the LIDAR. 

In the part of the path that is shown in the upper right quadrant in Figure~\ref{fig:cylinder_result}, as soon as the robot has lost the first cylinder from its field of view, only the middle cylinder is seen for sometime until the robot turns and the third cylinder comes close enough to be resolved. This is the time interval when the SLAM reconstructed path deviates from the actual path as the heading was being previously corrected using the cylinders detected, but when the landmark is missed, the pure odometry keeps moving it in the same direction. A similar effect is seen in the upper right corner when the only cylinder that can be seen is the uppermost one. This compounds the error previously accumulated. It is only when the robot is able to see both the lower and the middle cylinders simultaneously, that the path is corrected. This overcompensation also results in a corresponding error in the reconstruction of the environment. This error as well as error in the end position of the robot is seen in Table~\ref{tab:cylinder_results}.

\begin{table}
\caption{Errors in environment reconstruction using point features.}
\label{tab:cylinder_results}
\begin{tabular}{| l | c | c | c |}
\hline ~ & Actual Position (x,y) & Detected Position (x,y) & Error(m)\\
\hline Cylinder 1 & (-1.2192,1.524) & (-1.2525,1.4851) & 0.002622 \\ 
\hline Cylinder 2 & (-0.6096,3.3528) & (-0.7522,3.3942) & 0.022048 \\ 
\hline Cylinder 3 & (-1.2192,5.1816) & (-1.5308,5.2055) & 0.097665 \\ 
\hline End Position & (-0.4459,2.3715) & (-0.5917,2.4141) & 0.023072 \\
\hline 
\end{tabular} 
\end{table}

\subsection{Using Linear Features}
\label{sec: hough_results}

In the controlled environment of Figure~\ref{fig:overhead_path}, the linear features are the 4 walls enclosing the space. For wall detection, two algorithms are discussed in Section~\ref{sec:extraction}. The RANSAC based algorithm, when implemented in the way discussed is seen to be not robust in Figure~\ref{fig: ransac_bad}. When used in SLAM it generates a number of spurious landmarks which result in wrong data association and subsequent correction. This results in the reconstruction being highly erroneous. 

Hence the Hough transform based algorithm is used. As described in Section~\ref{sec:hough}, the only tuning parameters for that algorithm are the pixel and angular resolution and threshold for line detection. Since these are broad parameters any reasonable value may be chosen. Since we use an image of 1000~$ \times $~1000 pixels to represent 10~m~$ \times $~10~m space in the real world, the physical significance of the pixel resolution is the minimum distance in centimeters that walls need to be far from each other for them to be considered as 2 individual walls instead of one. The angular resolution is useful as it avoids finding multiple walls with small angles between them. This is especially useful while reconstructing indoor environments as most walls tend to intersect at right angles hence corresponding to a very low probability that the actual line features intersect at a small angle. In this arena, an angular resolution of $ 10^\circ $ and a distance resolution of 2~cm is chosen. This allows a good reconstruction while speeding up the Hough transform itself as the size of the accumulator from Equation~\ref{eq:hough3} is reduced. 

\begin{figure}
\centering
\includegraphics[height = 0.9\textheight]{walls}
\caption{Controlled environment and path reconstructed by detecting linear features in LIDAR data. Path starts from the origin.}
\label{fig:wall_result}
\end{figure}

The noise covariances for the encoder and LIDAR are kept the same as in Section~\ref{sec:Spike_results} since the same instruments are used. The only part that needs to be different from Section~\ref{sec: point_result} is the data association. There are a large number of ways that a linear feature can be associated. The simplest method is the Euclidean distance. This is feasible as the lines are represented in the normal form as shown in Figure~\ref{fig: normal_form}. Hence, the coordinates of the line are actually the point of intersection of the normal from the origin. If there is a difference in either angle or distance, the point of intersection will move. Hence the Euclidean distance between the detected and preexisting lines gives a good measure for association.

However, using solely the Euclidean distance is seen to have a drawback. It can lead to a wrong association when the lines are sufficiently close to the origin, such as in the bottom right corner of Figure~\ref{fig:overhead_path}. Then, the Euclidean distance between the points of intersection of the normals is lower than the threshold causing the 2 perpendicular walls to be associated as the same wall. Hence, it is necessary to add another heuristic to avoid this. The heuristic added in this implementation was based on the angle of intersection between the walls. The additional condition is that, even if Euclidean distance is small, if the angle of intersection is not close to 0 or 180 degrees, then the features are not associated with each other. With this addition, the path of the robot was reconstructed as shown in Figure~\ref{fig:wall_result}.


It is seen that the SLAM reconstruction is able to track the actual position for the most part except during the turn at the top and in a part close to the left wall. In the former case, the deviation from the path is mainly due to the odometry. This occurs as the robot has turned sufficiently to lose sight of the right wall and there are only small sections of the back and left wall in its field of view. This causes the walls to be not detected and emphasis is therefore, given to the odometry. In the part close to the left wall, the robot is moving slightly towards the wall in the ground truth whereas according to the odometry it is almost parallel. Since the only wall that the LIDAR can see at this point is the left wall and since the left wall has only recently been observed there is a greater tendency to move the estimated wall than the robot. This is also the reason in Figure~\ref{fig:wall_result}, for the left wall to be seen at an angle. Once the robot close enough to see the back wall, the SLAM reconstruction does a large correction resulting in the jump seen. The errors in the environment reconstruction as well as the path are shown in Table~\ref{tab:wall_results}. 

\begin{table}
\caption{Errors in environment reconstruction using linear features.}
\label{tab:wall_results}
\begin{tabular}{| l | c | c | c |}
\hline ~ & Actual Position (x,y) & Detected Position (x,y) & Error(m)\\
\hline Wall 1 & (0,-0.3048) & (-0.0066 -0.4371) & 0.017547 \\ 
\hline Wall 2 & (0.3048,0) & (0.3018,-0.0018) & 0.000012 \\ 
\hline Wall 3 & (0,6.096) & (-0.1784,6.3731) & 0.108611 \\ 
\hline Wall 4 & (-1.6764,0) & (-1.7358,-0.0447) & 0.005526 \\
\hline End Position & (-0.4459,2.3715) & (-0.5362,2.3477) & 0.008721 \\
\hline 
\end{tabular} 
\end{table}

\subsection{Using a Combination of Point and Linear Features}
\label{sec:combo_result}

As seen in Section~\ref{sec:slam_process}, the implementation of SLAM algorithm is designed to use both kinds of features simultaneously. With all the implementation details as in Sections~\ref{sec:Spike_results} and~\ref{sec: hough_results}, the reconstructed path is shown in Figure~\ref{fig:combo_result}.

\begin{figure}
\centering
\includegraphics[height = 0.9\textheight]{both}
\caption{Controlled environment and path reconstructed by detecting both point and linear features in LIDAR data. Path starts from the origin}
\label{fig:combo_result}
\end{figure}


From the path, it is seen that a few of the errors from the previous two reconstructions have been corrected. For example, in Figure~\ref{fig:cylinder_result}, there is a large error in the top left part of the plot due to the cylinder not being detected robustly. In this path, the lack of the  cylinder feature is compensated for by using the walls. Since it is a corner region the algorithm is able to correct both position and orientation. In Figure~\ref{fig:wall_result}, there is a jump in the path near the left wall, and the left wall is reconstructed at an angle, as discussed in Section~\ref{sec: hough_results}. When the robot is traveling parallel to the left wall, it has 2 cylinders in its field of view, and hence, this jump is avoided.

There are still deviations from the path near the left wall as both the linear features based and the point features based reconstructions had both deviated at that region, a combination of the two cannot yield an accurate reconstruction, though it can be better than either one individually. The environment reconstruction errors are tabulated in Table~\ref{tab:combo_results}.

\begin{table}
\caption{Error in environment reconstruction using combination of linear and point features.}
\label{tab:combo_results}
\begin{tabular}{| l | c | c | c |}
\hline ~ & Actual Position (x,y) & Detected Position (x,y) & Error(m)\\
\hline Cylinder 1 & (-1.2192,1.524) & (-1.2856,1.5376) & 0.004602 \\ 
\hline Cylinder 2 & (-0.6096,3.3528) & (-0.5768,3.3911) & 0.002543 \\ 
\hline Cylinder 3 & (-1.2192,5.1816) & (-1.1308,5.2717) & 0.015932 \\
\hline Wall 1 & (0,-0.3048) & (-0.0177,-0.3476) & 0.002145 \\ 
\hline Wall 2 & (0.3048,0) & (0.2805,-0.0080) & 0.000654 \\ 
\hline Wall 3 & (0,6.096) & (-0.3663,6.1225) & 0.134878 \\ 
\hline Wall 4 & (-1.6764,0) & (-1.7436,0.0152) & 0.004747 \\
\hline End Position & (-0.4459,2.3715) & (-0.5013,2.4103) & 0.00457 \\
\hline 
\end{tabular} 
\end{table}

To compare the reconstructions, we can compare the landmark error values from the tables. In Table~\ref{tab:combo_results}, the total of all the cylinder errors is 0.0231 m. The corresponding errors in Table~\ref{tab:cylinder_results} add up to 0.1223 m. Hence, in terms of finding the location of the cylinders using the combination of both kinds of features reduces the error by 81.1\%. Similarly The total error in walls, for the combination of both is 0.142 and with just linear features is 0.132. Hence, using the combination of features is seen to increase the error, but only by 8\%. If we consider the end position of the robot, we see that compared to the point features based reconstruction, there is a 80\% reduction of error in the reconstruction using both types of features and a 47.5\% reduction compared to using only linear features. 

\subsection{Uncontrolled Environment with Linear Features}

\begin{figure}
    \centering
    \begin{subfigure}[b]{0.45\textwidth}
	    \includegraphics[width=\textwidth]{corr1}
    \end{subfigure}
    \quad %add desired spacing between images, e. g. ~, \quad, \qquad, \hfill etc.
      %(or a blank line to force the subfigure onto a new line)
    \begin{subfigure}[b]{0.45\textwidth}
        \includegraphics[width=\textwidth]{corr2}
    \end{subfigure}%

    \caption{Passages forming an uncontrolled environment.}
    \label{fig:onboard_1}
\end{figure}

\begin{figure}
    \centering
    \begin{subfigure}[b]{0.45\textwidth}
	    \includegraphics[width=\textwidth]{corr3}
    \end{subfigure}
    \quad %add desired spacing between images, e. g. ~, \quad, \qquad, \hfill etc.
      %(or a blank line to force the subfigure onto a new line)
    \begin{subfigure}[b]{0.45\textwidth}
        \includegraphics[width=\textwidth]{corr_human}
    \end{subfigure}%

    \caption{Obstacles in an uncontrolled environment.}
    \label{fig:onboard_2}
\end{figure}
The primary purpose of implementing EKF SLAM is to run it in real time. For this, it is essential that the tuning parameters previously discussed should not need to be tuned for each specific arena configuration. Hence, a longer run in an uncontrolled environment  with the same tuning parameters is analyzed.

From Section~\ref{sec:combo_result}, it is clear that the linear features based SLAM is relatively accurate compared to point features based SLAM. Although the combination of both walls and cylinders for SLAM gave the best results, the feature extractor for cylinder is seen to be conservative and not entirely robust. In Section~\ref{sec:hough}, it was shown that the Hough transform based linear feature extractor is robust to humans in its field of view. Hence for a long range uncontrolled indoor environment, the linear feature based SLAM is an ideal candidate. 

To test this, the robot is driven around a long passage which has both open and closed doors on either side as seen in Figure~\ref{fig:onboard_1}. There are also people walking around and entering and leaving through those doors. As seen in Figure~\ref{fig:onboard_2}, there are also other unintentional point features in the environment. 

This gives a challenge for both EKF based SLAM and for the Hough feature extractor. The correction should ideally be made based only on walls and not other lines such as door ways etc. For SLAM itself, since the distance traveled is much larger and the velocity with which the robot moves is also larger, the encoders drift by a considerable amount as seen in Figure~\ref{fig:big_run}. Since the run is in a larger environment, the ground truth is approximated manually, without any overhead camera data. It can be seen that linear feature based SLAM does a good job in estimating the path traveled compared to the odometry. It does not completely reconstruct the ground truth due to the large number of disturbances throughout it's path such as doors. Towards the end of the run, a number of estimated wall are seen due to the shape of the passages in that region. 

\begin{figure}
\centering
\includegraphics[height = 0.9\textheight]{big_run}
\caption{Uncontrolled environment and path reconstructed by detecting linear features in LIDAR data. Path starts from the origin.}
\label{fig:big_run}
\end{figure}

Hence, it is seen that in an uncontrolled environment, the Hough transform based linear feature detector performs well at reconstructing the path of the robot, even though the reconstruction of the environment is not accurate. It is also seen that the tuning parameters chosen in Sections~\ref{sec:Spike_results} and~\ref{sec: hough_results}, are robust and do not need to be changed for similar environments.  
\chapter{Conclusion}
\label{cha:conclusion}
This thesis presented an implementation of Simultaneous Localization and Mapping on the Integrated Mobile Platform. A general overview of EKF and its application to SLAM was given. The Integrated Mobile Platform, a differential drive platform with capabilities of real-time implementation of SLAM was designed and built. The platform was equipped with encoders for odometry, a LIDAR and a camera for environmental sensing. A multi level system architecture was set up. The motion models for the platform and how they are used in EKF was discussed. Two kinds of features were detected from the LIDAR data: 1) Point features which consisted of cylinder like objects placed in the arena, and 2) A linear feature extractor which was used to reconstruct the walls of the arena. A simplistic algorithm for point features was shown to be not robust enough to be used in a real world applications. Modifications to the algorithms were proposed. For linear features, two algorithms were discussed, one based on RANSAC and one based on Hough transform. The Hough transform based algorithm was shown to be robust even to humans moving through the environment. Both the linear and point features were utilized in the SLAM algorithm and results were presented. The reconstructed paths were analyzed to obtain intuition about the process. It was seen that in a controlled environment, the combination of point features and linear features give the least error. In an uncontrolled environment with people moving across the field of view of the LIDAR, only the Hough transform is robust enough to be used and it is seen to reconstruct the path with reasonable accuracy. 


Further work can include more robust feature extraction for point features by utilizing the optimized functions of OpenCV. Robust feature extractors such as sift or surf should also be considered and experimented with. Also in this thesis the on-board camera available on the platform was not utilized. Further work on analyzing the images acquired by the on-board camera can result in augmenting encoder based odometry by visual odometry. With visual odometry, the SLAM implementation will not be dependent on the robot having encoders and can be used for a variety of platforms. Since the platform is equipped with a real time arm processor, implementing the current SLAM algorithm in real-time will provide more insight about the different feature extractors discussed in terms of their speed and processing power requirements. The computational power on the platform is capable of running more complex autonomous navigation algorithms. The maps obtained by SLAM can be used as a starting point for algorithms such as path planning. Lastly,sensor fusion for LIDAR and camera to detect and discern features togrther are an important area of future work.


\bibliographystyle{IEEEtran}
\bibliography{IEEEabrv,bibliography}

%\include{bib}
\end{document}