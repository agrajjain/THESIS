\chapter{Hough transform based linear feature extractor}

\section{Overview}

In this implementation we use the LIDAR described in section ... as the input for the measurement $ z $ discussed in section \ref{sec:EKF_SLAM}. Both the measurement model and data association is derived based on there being linear features in the environment. Instead of a custom implementation which has a large number of tuning parameters we leverage the existing implementation of Hough transform on OpenCV and find the lines and represent them in the normal form. The measurement model is derived based on this form.


\section{Feature Extraction Algorithm}
\textit{More description of what is Hough transform. HOUGH feature extraction}

\section{Measurement Model and the corresponding differentials}
\textit{The mathematical model for measurement.}

\section{Experimental results}
\textit{Description of the arena and the run performed.}

\textit{Images of the path ground truth and correction.}

\textit{It gives good estimate of walls and is much faster.}
 