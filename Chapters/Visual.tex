\chapter{Visual odometry}

\section{Overview}

Traditionally a Camera is used as and exteroceptive sensor to get measurements of the environment. But it is interesting how it can also be used as a proprioceptive sensor using visual odometry\cite{}. By this, the number of sensors on a robot can be reduced saving cost and weight which are of extreme importance in certain cases. 


\section{Conceptual Description}
\textit{Description of epipolar geometry}
\textit{The mathematical base for visual odometry}

\section{Implementation and integration with EKF SLAM}
\textit{Visual odometry based motion model and Jacobian}

\section{Experimental results}
\textit{Description of the arena and the run performed.}
\textit{Images of the path ground truth and prediction.}
\textit{Images of the path ground truth and correction.}

Not as good. But good enough. Shown using corrected path
 